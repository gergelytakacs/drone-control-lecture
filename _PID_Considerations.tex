

\begin{frame}[t]{Inicializácia riadenia}
\begin{itemize}
  \item<1-> Ako naštartujeme riadenie? Aká je odchýlka $e(0)$ pri štarte?
  \item<2-> Kritické pri zmene letových módov \citep{AP:PID,Bresciani2020}
  \item<3-> Pre ArduCopter \citep{AP:PID}
    \begin{itemize}
    \item Nadradené riadenie poloha vs. rýchlosť (P) --- tak aby vstup bol konštantný
    \item Rýchlosť (PID) --- $e(0)=0$, I zložka s konštantným vstupom, zrátať D pri nastavení žiadanej hodnoty
\end{itemize}

\end{itemize}
\end{frame} 


%\begin{frame}
%Notes: \cite{AP:PID}
%User (ROS) $\rightarrow$ Shaping $\rightarrow$ PID $\rightarrow$ Actuators
%Yaw je prioritizovanych nad Pitch Roll, lebo to drzi dron v lufte \citep{Erasmus2020}
%50 Hz -> 400 Hz
%Min 24 tazsie uchopytelne koncepty pre prezentaciu, dava menej konkretnosti
%Velocity prioritizuje vertikalnu rychlost   \citep{Erasmus2020}
%\end{frame}

%\begin{frame}[t]{Others}
%\begin{itemize}
%  \item<1-> Gain scheduling pre Heli\citep{Hall2020}
%\end{itemize}
%
%Actuator output proportional then FF is good
%Copter doesnt have FF in rate
%Dynamic Notch filter na zavislosti od RPM propellerov (vibracie vs. snimace). Rozmyslaju aj realtime FFT
%Akrobatika - pri otacani vypne Yaw a sustredi na trhust vector (ked sme hore nohami, naco riesit yaw)
%System Identification Mode - Chirp inject v AP, dev dufa ze univerzity sa pripoja!
%
%Yaw je rychlejsie ako Pitch a Roll, preto Yaw slucka ma LP pre cely PID nielen D
%Trigonometric S-Curve spline pre polohovanie/posunutie - vyhladenie reakcie na sledovanie WP v rohu - je to čiastočne aj o tom, aby to vyzeralo dobre a zvýšil istotu riadenia\footnote{inspire confidence...} oz strany uživateľa. Odsráni náhle zmeny, tzv. trhanie \angl{twitch} cez zmenu profilu derivácie lineárneho zrýchlenia - trhania \angl{jerk}\footnote{dvojitá derivácia zrýchlenia je \angl{snap} alebo \angl{jounce}}.
%\end{itemize}
%\end{frame} 

% Model-based ladenie!


%Other interesting ones
%https://www.youtube.com/watch?v=4hlQ2pf842U
%https://www.youtube.com/watch?v=iS5JFuopQsA
%https://www.youtube.com/watch?v=7F9cG64kRxI
%https://www.youtube.com/watch?v=fpJZSQmqDVk