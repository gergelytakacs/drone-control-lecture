

\begin{frame}[t]{Inicializácia riadenia}
\begin{itemize}
  \item<1-> Ako naštartujeme riadenie? Aká je odchýlka $e(0)$ pri štarte?
  \item<2-> Kritické pri zmene letových módov \citep{AP:PID,Bresciani2020}
  \item<3-> Pre ArduCopter \citep{AP:PID}
    \begin{itemize}
    \item Nadradené riadenie poloha vs. rýchlosť (P) --- tak aby vstup bol konštantný
    \item Rýchlosť (PID) --- $e(0)=0$, I zložka s konštantným vstupom, zrátať D pri nastavení žiadanej hodnoty
\end{itemize}
\end{itemize}
\end{frame}


\begin{frame}{Prioritizácia slučiek}
\begin{itemize}
  \item<1-> Klopenie a klonenie je najdôležitejšie, to drží dron v lete \citep{AP:PID,Erasmus2020}
  \item<2-> Pri agresívnych manévroch (akrobatika) slučka vybočenia môže byť úplne vypnutá: prečo by sme riešili vybočenie ak sme obrátení na hlavu? \citep{Hall2020}
  \item<3-> Pre riadenie rýchlosti prioritizujeme vertikálnu rýchlosť \citep{Erasmus2020}
  \item<4-> Z pohľadu dynamiky manéver vybočenia je rýchlejšia ako klopenie a vybočenie. Je to jeden z dôvodov prečo má táto slučka low-pass filtrovanie pre všetky komponenty PID, nielen D \citep{Hall2020}.
\end{itemize}
\end{frame}



\begin{frame}{Možné problémy a vylepšenia}
\begin{itemize}
  \item<1-> Gain scheduling --- v AP plánované pre Heli \citep{Hall2020}
  \item<2-> Adaptívne regulátory --- AP už má tzv. ``System Identification Mode'' ktorý injektuje chirp signál pre offline ID \citep{Hall2020}. Hľadajú spoluprácu univerzít pre podobné hlbšie témy \citep{Hall2020}!
  \item<3-> Ladenie PID na základe modelu? (nelineárny model. vs. linearizovaný)
\end{itemize}
\end{frame}


\begin{frame}{Zaujímavé novinky v AP}
\begin{itemize}
  \item<1-> AP už má dynamický zárezový filter \angl{dynamic notch filter} filter, ktorý filtruje (aj) vyššie harmonické frekvencie na zavislosti od RPM propellerov na základe FFT \citep{Hall2020}! (samozrejme nechceme vibrácie motorov, airframe snímať...)
      \item<2-> Trigonometrické S-krivky \angl{trigonometric S-Curve} pre polohovanie/posunutie - slúži na vyhladenie reakcie na sledovanie WP napr. v rohu. Je to čiastočne aj o tom, aby to vyzeralo dobre a zvýšil istotu riadenia\footnote{``inspire confidence...'' \citep{Hall2020}} zo strany užívateľa. Odstráni náhle zmeny, tzv. trhanie \angl{twitch} cez zmenu profilu derivácie lineárneho zrýchlenia - trhania \angl{jerk}\footnote{dvojitá derivácia zrýchlenia je \angl{snap} alebo \angl{jounce}}.
\end{itemize}
\end{frame}


%User (ROS) $\rightarrow$ Shaping $\rightarrow$ PID $\rightarrow$ Actuators
%Actuator output proportional then FF is good
%Copter doesnt have FF in rate





%Other interesting ones
%https://www.youtube.com/watch?v=4hlQ2pf842U
%https://www.youtube.com/watch?v=iS5JFuopQsA
%https://www.youtube.com/watch?v=7F9cG64kRxI
%https://www.youtube.com/watch?v=fpJZSQmqDVk 